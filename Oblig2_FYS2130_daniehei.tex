\documentclass[a4paper,norsk, 10pt]{article}
\usepackage[utf8]{inputenc}
\usepackage{verbatim}
\usepackage{listings}
\usepackage{graphicx}
\usepackage[norsk]{babel}
\usepackage{a4wide}
\usepackage{color}
\usepackage{amsmath}
\usepackage{float}
\usepackage{amssymb}
\usepackage[dvips]{epsfig}
\usepackage[toc,page]{appendix}
\usepackage[T1]{fontenc}
\usepackage{cite} % [2,3,4] --> [2--4]
\usepackage{shadow}
\usepackage{hyperref}
\usepackage{titling}
\usepackage{marvosym }
\usepackage{subcaption}
\usepackage[noabbrev]{cleveref}
\usepackage{cite}


\setlength{\droptitle}{-10em}   % This is your set screw

\setcounter{tocdepth}{2}

\lstset{language=c++}
\lstset{alsolanguage=[90]Fortran}
\lstset{alsolanguage=Python}
\lstset{basicstyle=\small}
\lstset{backgroundcolor=\color{white}}
\lstset{frame=single}
\lstset{stringstyle=\ttfamily}
\lstset{keywordstyle=\color{red}\bfseries}
\lstset{commentstyle=\itshape\color{blue}}
\lstset{showspaces=false}
\lstset{showstringspaces=false}
\lstset{showtabs=false}
\lstset{breaklines}
\title{FYS2130 Oblig 2}
\author{Daniel Heinesen, daniehei}
\begin{document}
\maketitle


\section*{Diskusjonsspørsmål.}

\section{3)}

Q-verdien er gitt ved 

$$
Q = \frac{f_0}{\Delta f}
$$

hvor $f_0$ er resonansfrekvensen, og $\Delta f$ er halvverdibredden. Jo høyere $f_0$ er jo høyere er Q-verdien, men jo større $\Delta f$ er jo mindre er den. Systemer med høyere resonansfrekvens vil kunne ha litt større $\Delta f$ og likevel ha en høy Q-verdi, sammenlikned med systemer med lav resonansfrekvens, hvor $\Delta f$ må være lav for å få en høy Q-verdi.


\section*{4)}
Vi kan skrive om slik at 

$$
\Delta f = \frac{f_0}{Q}
$$

Om vi ønsker å kunne bedre skille mellom lyder på $\Delta f$ bli mindre. Gitt at Q-verdien holder seg konstant, så må derfor $f_0$ bli lavere. Vi ville derfor ikke kunne høre lyder men høy frekvens. 

\section*{5)}

\textbf{FYLL INN MER HER!}

De to resonanstypene kan sammenfalle, men bare vist man ikke har noe dempning i systemet.

\section*{6)}
Amplituden til et svingesystem er gitt ved

$$
A = \frac{F/m}{\sqrt{(\omega_0^2 - \omega_F^2)^2 + (\omega_Fb/m)^2}}
$$

Vi ser at om det ikke er noe dempning, så vil $b = 0$. Og om påtrykket har resonansfrekvens vil $(\omega_0^2 - \omega_F^2)^2 = 0$. Da vil amplituden gå mot uendelig.\\

Om påtrykksfrekvensen er litt forskjellig fra resonansfrekvensen vil $(\omega_0^2 - \omega_F^2)^2$ være veldig liten, men ikke null. Så amplituden vil være VELDIG stor, men ikke uendelig.



\section*{Regneoppgaver.}

\section*{12)}

Q-faktoren kan defineres som

\begin{equation}
Q = \frac{f_0}{\Delta f}
\label{Q}
\end{equation}

hvor $f_0$ er resonansfrekvensen, og $\Delta f$ er halvverdibredden. For å kunne få inn radiokanalen trenger kretsen en resonansfrekvens som er lik frekvensen radiostasjonen. Så $f_0 = 1313 $kHz. Vi vil også at resonansfrekvensen til kretsen ikke skal kunne nærme seg frekvensen til en annen radiostasjons. Siden de er avskilt med minst $9$kHz setter vi $\Delta f = 9$kHz. Vi får da at kretsens Q-verdi skal være

$$
Q = \frac{1313}{9} = 145.9
$$

\section*{13)}

\subsection*{a)}
Fra (3.15) vet vi at

$$
\Delta t = \frac{QT}{2\pi} = \frac{Q}{2\pi f}
$$

Dette er tiden det tar før lydbølgen er $1/e$ av initial energien. Vi vil at lyden flaggermusen lager skal rekke å sprette i veggen og kommet tilbake til flaggermusen før dette skjer. Vi vet at lyden har en fart på $v = 340m/s$, vi kan ganne $\Delta t$ med dette for å få en avstand.

$$
s = \frac{Qv}{2\pi f}
$$

Men dette er bare en vei, men lyden må gå fra flaggermusen til veggen, og så tilbake igjen. Denne veien $S$ er $S = 2s$

$$
S = \frac{Qv}{\pi f}
$$

Setter vi inn frekvene flaggermusen kan lage $f = [40 kHz,100 kHz]$, får vi at lyden har gått $S = [0.11m,0.27m]$. Avstanden til selve veggen er halvparten av dette. Så minste avstanden til veggen er $0.55$m

\subsection*{b)}

Vi kan sette inn $f = 1000$ Hz.

$$
S = \frac{Qv}{\pi  f} = 10.8 \mathrm{m}
$$

Avstanden flaggermusen nå kan merke veggen er $5.4$m


\section*{18)}
\subsection*{a)}

Vi kan sammenlikne (3.7)

$$
L\frac{d^2Q}{dt^2} + R\frac{dQ}{dt} + \frac{1}{C}Q = V_0 \cos(\omega_F t)
$$

med (3.1)

$$
\ddot{z} + \frac{b}{m}\dot{z} + \omega_0^2 z = \frac{F}{m}\cos(\omega_F t)
$$

Vi ser at begge er differensiallikninger på samme form. Eneste forskjellen er konstantene. Sammenlikner vi disse kan vi se at

$$
\frac{b}{m} = \frac{R}{L}, \qquad \omega_0^2 = \frac{1}{LC},\qquad \frac{F}{m} = \frac{V_0}{L}
$$

og at 

$$
b = R, \qquad m = L, \qquad k = \frac{1}{C}
$$

Vi kan bruke disse relasjonene til å finne faseskiftet, amplituden og Q-verdien for serie-RCL-kretsen.

$$
\cot \phi = \frac{\omega_0^2 - \omega_F^2}{\omega_F \frac{b}{m}} = \frac{\frac{1}{LC} - \omega_F^2}{R \frac{V_0}{L}}
$$

$$
A = \frac{F/m}{\sqrt{(\omega_0^2 - \omega_F^2)^2 + (\omega_Fb/m)^2}} = \frac{V_0/L}{\sqrt{(\frac{1}{LC} - \omega_F^2)^2 + (\omega_FR/L)^2}}
$$

$$
Q = \sqrt{\frac{mk}{b^2}} = \sqrt{\frac{L}{R^2C}}
$$

\subsection*{b)}

Vi har fått oppgit at for denne kretsen er 	$R = 1.0$ohm, $C = 100$nF og $L = 25 \mu \mathrm{H}$.\\

Faseresonansen til kretsen er 

$$
f_{fase.res.} = \frac{1}{2\pi}\omega_0 = \frac{1}{2\pi}\sqrt{\frac{1}{LC}} = 100.7 \mathrm{kHz}
$$

Ampituderesonansen er gitt ved 

$$
f_{amp.res.} = \frac{1}{2\pi}\sqrt{\omega_0^2  - \frac{b^2}{2m^2}} = \frac{1}{2\pi}\sqrt{\frac{1}{LC}  - \frac{R^2}{2L^2}} = 100.6 \mathrm{kHz}
$$

\subsection*{c)}

Vi kan finne Q-verdien med

$$
Q  = \sqrt{\frac{L}{R^2C}} = 15.8
$$

\subsection*{d)}
Ved faseresonanse er

$$
f_{fase.res.} = \frac{1}{2\pi}\omega_0
$$

Ganger vi dette med $2\pi$ får vi

$$
2\pi f_{fase.res.}  = \omega_{fase.res} = \omega_0
$$

Setter vi dette inn for $\omega_F$ får vi at 

$$
\omega_0^2 - \omega_F^2 = 0
$$

Ser vi på uttrykket for faseskifte ser vi at 

$$
\cot \phi = 0
$$

Dette gir oss en faseforskjell på $\pi /2$.\\

Om vi bruker at 

$$
\omega_F = \omega_0 + \frac{\Delta \omega}{2} = \omega_0 + \frac{\omega}{2Q}
$$

Får vi at 

$$
\omega_0^2 - \omega_F^2 = -\frac{5\omega^2}{4Q}
$$

Da blir 

$$
\cot \phi = -\frac{5L}{4\sqrt{LC}(1+\frac{1}{LC})}
$$

Da er faseforskjellen

$$
\phi = - 0.6896
$$


\end{document}


