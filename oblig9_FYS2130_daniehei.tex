\documentclass[a4paper,norsk, 10pt]{article}
\usepackage[utf8]{inputenc}
\usepackage{verbatim}
\usepackage{listings}
\usepackage{graphicx}
\usepackage[norsk]{babel}
\usepackage{a4wide}
\usepackage{color}
\usepackage{amsmath}
\usepackage{float}
\usepackage{amssymb}
\usepackage[dvips]{epsfig}
\usepackage[toc,page]{appendix}
\usepackage[T1]{fontenc}
\usepackage{cite} % [2,3,4] --> [2--4]
\usepackage{shadow}
\usepackage{hyperref}
\usepackage{titling}
\usepackage{marvosym }
\usepackage{subcaption}
\usepackage[noabbrev]{cleveref}
\usepackage{cite}


\setlength{\droptitle}{-10em}   % This is your set screw

\setcounter{tocdepth}{2}

\lstset{language=c++}
\lstset{alsolanguage=[90]Fortran}
\lstset{alsolanguage=Python}
\lstset{basicstyle=\small}
\lstset{backgroundcolor=\color{white}}
\lstset{frame=single}
\lstset{stringstyle=\ttfamily}
\lstset{keywordstyle=\color{red}\bfseries}
\lstset{commentstyle=\itshape\color{blue}}
\lstset{showspaces=false}
\lstset{showstringspaces=false}
\lstset{showtabs=false}
\lstset{breaklines}
\title{FYS2130 Oblig 9}
\author{Daniel Heinesen, daniehei}
\begin{document}
\maketitle

\section*{Oppgave 12)}
Vi kan bruke Snells love

\begin{equation}
n_1 \sin \theta_1 = n_2 \sin\theta_2
\label{eq:snell}
\end{equation}

Hvor $n_1 = 1.333$ er brytningsideksen til vannet, $\theta_1$ er innfallsvinkelen til lyset på 'vannsiden'. $n_2$ er brytningsindeksen til glasset og $\theta_2 = 48.7^{\circ}$ er utfallsvinkelen til lyset. Om $\theta_1 = \theta_c$ er den kritiske vinkelen, så vil lysstrålen i glasset bøyes og gå parallelt med overflaten på glasset, hvilket betyr at $\theta_2 = \frac{\pi}{2}$. Fra snells lov \eqref{eq:snell} får vi da

$$
n_1 \sin \theta_1 = n_1 \sin \theta_c = n_2 \sin\theta_2 = n_2 \sin\frac{\pi}{2} = n_2
$$
$$
\Rightarrow n_2 = n_1 \sin \theta_c
$$

Vi kan så finne brytningsindeksen til glasset:

\begin{equation}
n_2 = 1.333\cdot\sin(48.7^{\circ}) = \underline{\underline{1.001}}
\end{equation}

\section*{Oppgave 14)}

Vi sender inn upolarisert lys mot et glassvindu. Det reflekterte lyset er perfekt polarisert, hvilket betyr at innfallsvinkelen må være\textit{Brewster}-vinkelen. Vi vet Brewster-vinkelen $\theta_B$ er gitt som

\begin{equation}
\theta_B = \arctan\left(\frac{n_2}{n_1}\right)
\label{eq:brewster}
\end{equation}

$n_1 \approx 1$ er brytningsindeksen til luften, mens $n_2$ er den ukjente brytiningsindeksen til glasset. Vi vet at innfallsvinkelen til lyset er $\theta_B = 54.5^{\circ}$, og vi kan bruke dette til å finne $n_2$:

$$
n_2 = n_1\tan \theta_B = \tan 54.5^{\circ} = \underline{\underline{1.402}}
$$

Vi vet også at for Brewster-vinkelen er det reflekterte lyset polarisert normalt på innfallsvinkelen. M.a.o: $R_p = 0$. Om vi bruker formelen for $R_p$ og løser for $n_2$, vil vi også komme frem til at $n_2 = 1.402$.\\

Vi kan også finne vinkelen til det transmitterte lyset. Vi vet av ved Brewster-vinkelen er

$$
\theta_r + \theta_t = 90^{\circ}
$$

Hvor $\theta_r$ er vinkelen til det reflekterte lyset, og $\theta_t$ er vinkelen til det transmitterte lyset. Vi vet også at det reflekterte lyset har samme vinkel som innfallsvinkelen. Dette gir oss at

$$
\theta_t = 90^{\circ} - \theta_r = 90^{\circ} - \theta_B = 90^{\circ} - 54.5^{\circ} = \underline{\underline{35.5^{\circ}}}
$$


\section*{Oppgave 16)}

Vi skal finne intensiteten til lys som går igjennom linære polarisasjonsfiltre. Forholdet mellom den orginale intensiteten og den intensiteten lyset har etter den har gått igjennom filteret er

\begin{equation}
I = I_0\cos^2(\theta_2 - \theta_1)
\label{eq:filter}
\end{equation}

\subsection*{a)}
Lyset i denne deloppgaven går først gjennom et filter som er dreid $\theta_1 = 15^{\circ}$, og deretter et filter som er dreid $\theta_2 = -70^{\circ}$. Lyset starter helt upolarisert, og etter den har gått igjennom første filter vil bare halvparten av intensisteten være igjen. Vi sier at vi starter med en intensiset på $1$ før lyset når filteret. Vi har da $I_0 = 0.5$ etter første filter. Dette gir en forskjell i intensiteten på

$$
I = 0.5\cos^2(-70^{\circ} - 15^{\circ}) = \underline{\underline{0.0038}}
$$

Vi ser at intensiteten bare er rundt $0.4 \%$ av det den var før første filter.


\subsection*{b)}
Vi setter nå et filter til mellom de to andre filterne. Den har en dreining på $\theta_3 = -32^{\circ}$. Vi finner først hvilke intensistet lyset har etter den har gått igjennom dette filteret

$$
I' = 0.5\cos^2(-32^{\circ} - 15^{\circ}) = 0.2326
$$ 

Lyset går så igjennom det neste filteret. Lyset vil etter dette filteret ha intensitet

$$
I = I'\cos^2(-70^{\circ} + 32^{\circ}) = \underline{\underline{0.1444}}
$$

Dette er betydelig mer enn når vi bare hadde to filtere.

\subsection*{c)}

Vi prøver nå å sette dette tredje filtere etter de to andre. Vi vet at \eqref{eq:filter} sier at filtre bare kan redusere intensiteten, så vi forventer en enda laver intensitet enn i \textit{a)}:

$$
I = 0.0038\cos^2(-32^{\circ} + 70^{\circ}) = \underline{\underline{0.00236}}
$$

Vi kan se at intensiteten nå bare er $0.24\%$ av den orginale intensiteten. Og som forventet er den lavere enn i \textit{a)}.



\end{document}


