\documentclass[a4paper,norsk, 10pt]{article}
\usepackage[utf8]{inputenc}
\usepackage{verbatim}
\usepackage{listings}
\usepackage{graphicx}
\usepackage[norsk]{babel}
\usepackage{a4wide}
\usepackage{color}
\usepackage{amsmath}
\usepackage{float}
\usepackage{amssymb}
\usepackage[dvips]{epsfig}
\usepackage[toc,page]{appendix}
\usepackage[T1]{fontenc}
\usepackage{cite} % [2,3,4] --> [2--4]
\usepackage{shadow}
\usepackage{hyperref}
\usepackage{titling}
\usepackage{marvosym }
\usepackage{subcaption}
\usepackage[noabbrev]{cleveref}
\usepackage{cite}
\usepackage{amsmath,mathtools}

\usepackage{fancyhdr}
\pagestyle{fancy}
\lhead{}
\chead{}
\fancyhead[R]{\textbf{Kadnr.: 15070}}
\fancyhead[L]{\textbf{FYS2130}}

\lfoot{}
\cfoot{}


\fancyfoot{} % clear all footer fields
\fancyfoot[LE,RO]{\thepage}


\renewcommand{\headrulewidth}{0.4pt}
\renewcommand{\footrulewidth}{0.4pt}


\newcommand{\fancyfootnotetext}[2]{%
  \fancypagestyle{dingens}{%
    \fancyfoot[L]{\parbox{12cm}{\footnotemark[#1]\footnotesize #2}}%
  }%
  \thispagestyle{dingens}%
}



\setlength{\droptitle}{-10em}   % This is your set screw

\setcounter{tocdepth}{2}

\lstset{language=c++}
\lstset{alsolanguage=[90]Fortran}
\lstset{alsolanguage=Python}
\lstset{basicstyle=\small}
\lstset{backgroundcolor=\color{white}}
\lstset{frame=single}
\lstset{stringstyle=\ttfamily}
\lstset{keywordstyle=\color{red}\bfseries}
\lstset{commentstyle=\itshape\color{blue}}
\lstset{showspaces=false}
\lstset{showstringspaces=false}
\lstset{showtabs=false}
\lstset{breaklines}
\title{FYS2130 Prosjekt}
\author{Kadnr.: 15070}
\begin{document}
\maketitle
\newpage

\section{Exercise 1)}

We are here going to find the expression for a mass point $y_i$ of a string from its neighbours and previous position. From the exercise text we are given the force on the a mass point as

\begin{equation}
F_i = F_{i,l} + F_{i,r} = -(k_{i-1} + k_i)y_i + k_{i-1}y_{i-1} + k_iy_{i+1}
\label{eq:Fi}
\end{equation}

Where $F_{i,l}$ is the force acting on the mass point from the left, and $F_{i,r}$ is the force acting on it from the right. $k_i$ is the spring constant of the $i$'th spring; $y_{i-1}$ and $y_{i+1}$ are the left and right neighbour, respectively; and $k_{i-1}$ the $i-1$'th spring.\\

From Newton's second law we get that

\begin{equation}
F_i = m_i\ddot{y_i}
\label{eq:ma}
\end{equation}

And from the exercise text we lastly get the numerical approximation of the second derivative:

\begin{equation}
\ddot{y_i} = \frac{d^2y_i}{dt^2} \approx \frac{y_i^+ -2y_i^0 + y_i^-}{(\Delta t)^2}
\label{eq:ydd}
\end{equation}

Where $y_i^+$ is what are interested in finding, namely the position of the mass point $y_i$ in the next time step. $y_i^0$ is the current position of the mass point, and $y_i^-$ the previous position. We are only going to look at the $y$-position of each mass point. In reality the mass point would be able to move in the $x$-direction as well(and $z$-direction in a 3D case), but we are simplifying the system by only looking at the $y$-position. \\

We can now use the information above to find a expression for the position of a mass point in the next time step $y_i^+$. We begin by inserting equation \eqref{eq:Fi} and \eqref{eq:ydd} into \eqref{eq:ma}:

\begin{equation}
-(k_{i-1} + k_i)y_i + k_{i-1}y_{i-1} + k_iy_{i+1} = m_i\frac{y_i^+ -2y_i^0 + y_i^-}{(\Delta t)^2}
\end{equation}

We can now simply solve for $y_i^+$. We then get the expression for the next $y$-position/amplitude of the $i$'th mass point:

\begin{align}
\Aboxed{
y_i^+ = \frac{\Delta t^2}{m_i}\left(k_{i-1}y_{i-1} - (k_{i-1} +k_i)y_i + k_iy_{i+1}\right) + 2y_i^0 -y_i^-
}
\label{eq:yi+}
\end{align}

Notice that this equation is dependent on the current and previous position o the mass point, meaning that we need an initial position \textit{and} a pre-initial position. Finding the pre-initial conditions will depend on the behaviour we want the system to have, and will be discussed below for some of the later cases.\\

We also have to be careful of the neighbouring mass points. In the middle of the string this posses no problem, but and the ends we have to impose some special rules on the mass points. We are going to look at two methods of dealing with the endpoints:

\newpage

\subsection*{Open ends:}

For the open ends we want the mass points to only feel the force of one of its neighbours -- the right neighbour for the first point, and the left of the last point --. This means that we can write Newton's second law for the first and last mass point as:

\begin{equation}
m_0\ddot{y}_0 = F_{0,r} \qquad m_{N-1}\ddot{y}_{N-1} = F_{N-1,l}
\end{equation}

We can thus rewrite expression \eqref{eq:yi+} explicitly for the endpoints:

\begin{equation}
y_0^+ = \frac{\Delta t^2}{m_0}\left(-k_0y_0 + k_0y_{1}\right) + 2y_0^0 -y_0^-
\end{equation}

\begin{equation}
y_{N-1}^+ = \frac{\Delta t^2}{m_{N-1}}\left(k_{N-2}y_{N-2} - k_{N-2} y_{N-1} + \right) + 2y_{N-1}^0 -y_{N-1}^-
\end{equation}


\subsection*{Reflective ends:}

The second type of ends are the reflective ends. These are ends that don't move from their initial position. This means that every wave hitting this point will be reflected. There are two ways of implementing this. The first and simplest method is to just force the position of these mass points to be the initial condition:

\begin{equation}
y_{end}^+ = y_{end}^0 = y_{end}^- 
\end{equation}

The second method is more like reality. A total reflection will happen when a wave hits a domain with a very large impedance, given as 

\begin{equation}
z_i = \sqrt{k_im_i}
\end{equation}

or a wave on string. We can there for give the endpoints very large impedance. In our case we can give the point a mass $M_i >> m_i$. This will ensure a reflective end. But again we have to find a way to treat the neighbours at the endpoint. The way I did this in my simulation was to simply have open ends with mass $M_i$, this means that we don't have to think about the neighbours and we get reflective ends. So for the reflective ends $y_{end}^+$ is given as:

\begin{equation}
y_0^+ = \frac{\Delta t^2}{M_0}\left(-k_0y_0 + k_0y_{1}\right) + 2y_0^0 -y_0^-
\end{equation}

\begin{equation}
y_{N-1}^+ = \frac{\Delta t^2}{M_{N-1}}\left(k_{N-2}y_{N-2} - k_{N-2} y_{N-1} + \right) + 2y_{N-1}^0 -y_{N-1}^-
\end{equation}


\newpage

\section{Exercise 2)}

We are here going to show that the above way of describing a wave on a string reduces to the 1D wave equation.\\

We are going to start be introducing a mass density $\mu = m/\Delta x$ and the constant spring stiffness $\kappa = k\Delta x$, here $\Delta x$ is the distance between the mass points. We are going to assume the spring constants $k_i$ and masses $m_i$ to be constant along the whole spring \footnotemark[1]\fancyfootnotetext{1}{This is not necessary to for the end points, and especially for reflective ends. But for a solution of the wave equation, the imposed boundary conditions will ensure the right behaviour of the ends. We can therefore ignore the ends for the derivation of the wave equation.}. This means that we can use that

\begin{equation}
k = \frac{\kappa}{\Delta x}, \qquad m = \mu \Delta x
\end{equation}

We can now insert these into equation \eqref{eq:Fi} and \eqref{eq:ma}

\begin{equation}
m_i\ddot{y}_i = \mu \Delta x\ddot{y}_i = F_i =  \frac{\kappa}{\Delta x}( y_{i-1} -2y_i + y_{i+1})
\end{equation}

\begin{equation}
\Rightarrow  \mu \Delta x\ddot{y}_i = \frac{\kappa}{\Delta x}( y_{i-1} -2y_i + y_{i+1})
\end{equation}

In the expression for $F_i$ I pulled out $\kappa/\Delta x$ and rearranged the $y$-terms. We then move $\mu\Delta x$ to the right side, and get

\begin{equation}
\ddot{y}_i = \frac{\kappa}{\mu}\cdot\frac{y_{i-1} -2y_i + y_{i+1}}{(\Delta x)^2}
\label{eq:almostWave}
\end{equation}

If we remember back to how discretized the second derivative of $y$ with respect to time in \eqref{eq:ydd} we see that the last term on the right side look eerily similar. This is infact the discretized of the second derivative of $y$ with respect to $x$

\begin{equation}
\frac{d^2y_i}{dx^2} \approx \frac{y_{i-1} -2y_i + y_{i+1}}{(\Delta x)^2}
\end{equation}

We are going to use that this go to a equivalence as $\Delta x$ goes to zero. So if we insert this into \eqref{eq:almostWave} we get

\begin{equation}
\ddot{y}_i = \frac{\kappa}{\mu}\frac{d^2y_i}{dx^2}
\label{eq:stepBeforeString}
\end{equation}

This means that a wave on a string obeys the differential equation:

\begin{align}
\Aboxed{
\frac{d^2y_i}{dt^2} = v_B^2\frac{d^2y_i}{dx^2}
}
\end{align}

This is the 1D wave equation. Comparing this with \eqref{eq:stepBeforeString}, we get that a wave on a string propagate with the velocity $v_B$ given as

\begin{align}
\Aboxed{
v_B^2 = \frac{\kappa}{\mu}
}
\end{align}




\newpage
\section{Exercise 3)}

When we are simulating the wave, we need to ensure that the program is numerically stable. The requirement for this to be true is that 

\begin{equation}
\frac{\Delta x}{\Delta t} \geq v_B
\label{eq:convergenceCond}
\end{equation}

This comes form the Courant–Friedrichs–Lewy condition for converges of a numerical solution for partial differential equations like the wave equation. Broadly this condition says that the time step has to be smaller that the time it takes the wave to move the next mass point, which is equivalent to the above \eqref{eq:convergenceCond}.\\

To see what kind of limitations this means or our simulation, we are going to rewrite $v_B$ with $k$ and $m$:

\begin{equation}
v_B = \sqrt{\frac{\kappa}{\mu}} = \sqrt{\frac{k\Delta x}{m/\Delta x}} = \Delta x\sqrt{\frac{k}{m}}
\end{equation}

We insert this into \eqref{eq:convergenceCond} 

\begin{equation}
\frac{\Delta x}{\Delta t} \geq \Delta x\sqrt{\frac{k}{m}}  
\end{equation}

Doing the algebra we get the convergence condition:

\begin{align}
\Aboxed{
\Delta t \leq \sqrt{\frac{m}{k}}
}
\label{eq:maxDt}
\end{align}

As we can see, the convergence condition no longer involve $\Delta x$. If we look at the expression for $y_i^+$ \eqref{eq:yi+} we see that there is no mention of $\Delta x$, meaning that both the stability and precision of the simulation is wholly independent of $\Delta x$.\\

The conclusion is thus that while the value of $\Delta t$ is highly important for the accuracy of the simulation, $\Delta x$ is just a scaling factor which we are free to set as $1$ for our simulation.

\newpage


\section{Exercise 4)}

\begin{equation}
y_i^0
\label{eq:initSin}
\end{equation}

\section{Exercise 5)}

We are interested in finding the frequency of a given mass point, using the same initial conditions as in the last exercise. We are going to start by finding it analytically. We are going to start with one of the most basic property of a wave:

\begin{equation}
f\lambda = v_B = \Delta x\sqrt{\frac{k}{m}}
\end{equation}

This gives the frequency

\begin{equation}
f = \frac{\Delta x}{\lambda}\sqrt{\frac{k}{m}} = \frac{\mathcal{K}\Delta x}{2\pi}\sqrt{\frac{k}{m}}
\end{equation}

Where $\mathcal{K}$ is the wave number of the initial sin wave. From \eqref{eq:initSin} we see that

\begin{equation}
\mathcal{K} = \frac{7\pi}{(N-1)\Delta x}
\end{equation}

The $\Delta x$ comes from the need to have the correct units. This finally gives the analytical value for the frequency:

\begin{align}
\Aboxed{
f_{analytic} = \frac{7}{2(N-1)}\sqrt{\frac{k}{m}} = 0.393278\text{ Hz}
}
\end{align}

We are now going to reproduce this numerically. We want the simulation to run for 10 periods. We can use the expected value the frequency to find the number of time steps which corresponds to 10 periods

\begin{equation}
\text{num. of time steps} = 10\cdot \frac{1}{\Delta t}\cdot \frac{1}{f} \approx 5685
\end{equation}

Here i used $\Delta t = 0.1\cdot\sqrt{m/k}$. If we look at the position of the $100$'th mass point after 10 periods we get

\begin{figure}[H]
\centering
\includegraphics[scale=0.4]{opp5y99.png}
\caption{The position of the $100$'th mass point as a function of time. I have marked one period of the oscillation. We can use this to find the frequency.}
\label{fig:y99}
\end{figure}

We can see the motion of the $100$'th mass point resemble a sinusoidal motion, as we would expect. We can look at the time between two maxima of the oscillation, and from the inverse of this we find the frequency numerically. To get a more accurate result I used the mean value of all the 10 periods, and got

\begin{align}
\Aboxed{f_{numerical} = \frac{1}{\bar{T}} \approx 0.393059 \text{ Hz}}
\end{align}

This is very close to the result we got analytically.\\

The last method we are going to use to find the frequency is a Fast Fourier Transform (FFT). We are going to use \textit{numpy's} FFT. If we do FFT of the motion in \ref{fig:y99} we get the following frequency spectrum:

\begin{figure}[H]
\centering
\includegraphics[scale=0.4]{opp5FFT.png}
\caption{The frequency spectrum of the $100$'th mass point. We can see a spike around $0.4$, just where we expect to find our frequency. The reason we don't get a spike at just one frequency is due to our frequency resolution. As $\Delta t \rightarrow 0$ we would get a spike just at one frequency.}
\label{fig:FFTy99}
\end{figure}

If we ask python to find where we get a maximum in figure above, we get the frequency of the oscillation:

\begin{align}
\Aboxed{f_{FFT} = 0.393328 \text{ Hz}}
\end{align}\\

We have now gotten the frequency of the oscillation of $y_{99}(t)$ from three different methods:

\begin{table}[H]
\centering
\begin{tabular}{|c|l|l|}
\hline
& Frequency & Relative error from analytic\\\hline
Analytic & $0.393278$ & - \\\hline
Numerical & $0.393059$ & $0.06\%$ \\\hline
Analytic & $0.393328$ & $0.013\%$ \\\hline
\end{tabular}
\caption{Frequency given by the different method, and their relative error from the analytical frequency.}
\end{table}

We can see that both numerical methods got very close to the analytic answer. This small error may come from the discretization. As $\Delta t$ becomes smaller, this error will most likely also go towards zero. I would have expected that the numerical frequency and the frequency from the FFT would have been closer to one another, since their error both comes from the size of the time step.

\section{Exercise 7)}
In this exercise we are going to check if our simulation are preserving the total energy. Total energy is $E = KE + U$ where

\begin{equation}
KE = \sum_i \frac{1}{2}m_iv_i^2 \qquad U = \sum_i \frac{1}{2}k_i\Delta y^2
\end{equation}

With

\begin{equation}
\Delta y = y_{i+1} - y_i, \qquad i = 0,\ldots,N-2
\end{equation} 

We are not including the $N-1$'th mass point since we are counting springs, we say are connected to the right of a mass point, thus the last one has no spring.\\

The one thing we need to discretized is the velocity. To get the best precision we are going to use the symmetric difference quotient

\begin{equation}
(y^{0}_i)^{'} \approx \frac{y_{i}^+-y_{i}^-}{2\Delta t}
\end{equation}

Since this is dependent on both the previous and next position, this is only possible for $0 < t < T$. So we have to use different method at $t = 0$ and $t = T$. For $t = 0$ we have to use
\begin{equation}
(y^{0}_i)^{'} \approx \frac{y_{i}^+-y_{i}^0}{\Delta t}
\end{equation}
and for $t = T$ we have to use

\begin{equation}
(y^{0}_i)^{'} \approx \frac{y_{i}^0-y_{i}^-}{\Delta t}
\end{equation}

This makes our calculation of the total energy less reliable at these points. Lets look at the result:

\begin{figure}[H]
\captionsetup[subfigure]{position=b}
\centering
\subcaptionbox{Shows the kinetic, potential end total energy of the system. We see from the red line that total energy is more or less conserved.}{\includegraphics[scale = 0.3]{opp6TotE.png}}
\subcaptionbox{The end $t = T$. We can see a small jump in kinetic and total energy.\label{fig:energyEnd}}{\includegraphics[scale = 0.3]{opp6End.png}}
\end{figure}

As we see above we have with a time step $\Delta t= 0.1\cdot \sqrt{m/k}$ an almost constant total energy. If we look at a larger time step we would begin to see a small oscillation in the total energy:

\begin{figure}[H]
\centering
\includegraphics[scale=0.4]{opp6TotEt1}
\caption{Simulation with $\Delta t= \sqrt{m/k}$. It is difficult to see, but the total energy is oscillating with a small amplitude and with the same frequency as the potential and kinetic energy.}
\end{figure}

If we look back to \ref{fig:energyEnd} we see a small jump. This is most likely due to the different method of numerical differentiation used here, which is less precise than the method used elsewhere.\\

All in all we can see that the total energy is well preserved in our simulation.

\section{Exercise 8)}
We are going to make a triangle centred around mass point 30. This means the our system starts with a initial condition

\begin{equation}
y_i^0 =
\begin{cases}
\frac{i}{30} & 0 \leq i \leq 29\\
\frac{59-i}{30} & 30 \leq i \leq 58\\
0 & \text{else}
\end{cases}
\end{equation}

We are then going to make it move to the right. To do make this happen we have to find specific pre-initial conditions $y_i^-$. There are two different methods to make this happen. One method where we set $\Delta t$ to some specific value, where we can just center the triangle one mass point/grid point backwards for $y_i^-$. The second method uses interpolation to find have the triangle looked one time step back.

\subsubsection*{Method 1:}

We are going to start with the general equation for a wave travelling to the right $F(x - vt)$. Since we have a discretized system, we know that in one time step $\Delta t$, the wave has to have travelled some distance $n\Delta x$, where $n \in \mathbb{N}$. This gives us that

\begin{equation}
F(x - v\Delta t) = F(x - n\Delta x) \Rightarrow v\Delta t = n\Delta x
\end{equation}

Using the definition of $v$ we get

\begin{equation}
\sqrt{\frac{k}{m}}\Delta x \Delta t = n\Delta x \Leftrightarrow \Delta t = n\sqrt{\frac{m}{k}}
\label{eq:initialTriangle}
\end{equation}

Comparing this to the convergence condition \eqref{eq:maxDt} we see that we must have $n = 1$. So we get that

\begin{align}
\Aboxed{
\Delta t = \sqrt{\frac{m}{k}}
}
\end{align}

For this specific value of $\Delta t$(the largest possible) we can just center the triangle one mass point to the left to get a triangle moving right.

\subsection*{Method 2:}
If we instead want a method which we can use for every possible value of $\Delta t$ we have to use interpolation. During one time step the a mass point a moved some $\Delta y$ either up or down(or not at all). We can use dimensional analysis to find the value of $\Delta y$. We take the derivative of the function of the initial condition with respect to the number of mass points(\eqref{eq:initialTriangle} is dependent on $i$, not $x$), $y^{'}$, we know that this is the amount the triangle has move from one mass point to the next $\Delta i$\footnotemark[2]\fancyfootnotetext{2}{$\Delta i$ is trivially equals to $1$, but is used here as a useful marker from the dimensional analysis}

\begin{equation}
y^{'} = \frac{\Delta y}{\Delta i}
\end{equation}

The velocity $v_B$ can be written as $v_B = \sqrt{k/m}\Delta x$ meaning that $\sqrt{k/m}$ is the velocity the wave moves from one mass point to the next during a time step.

\begin{equation}
\sqrt{\frac{k}{m}} = \frac{\Delta i}{\Delta t}
\end{equation}

If we combine these we get that

\begin{equation}
\Delta y = y^{'}\sqrt{\frac{k}{m}}\Delta t
\end{equation}

And from this we get the function for the pre-initial conditions:

\begin{align}
\Aboxed{
y_i^- = y_{i}^0 - \Delta y_{i+1}  = y_{i}^0 - y^{'}_{i+1}\sqrt{\frac{k}{m}}\Delta t
}
\end{align}

Notice that this is dependent on $y^{'}_{i+1}$, the derivative one step to the right.
We can so look at the results:

\subsection*{Results Method 1:}

For the following results I have chosen to center the triangle at $31$ instead of $30$, due to have to move the triangle one mass point to the left for the pre-initial conditions. I have also done this in the \textit{method 2}-results, to have comparable results for the two methods

\begin{figure}[H]
\captionsetup[subfigure]{position=b}
\centering
\subcaptionbox{The initial triangular wave.}{\includegraphics[scale = 0.3]{opp8t0M1.png}}
\subcaptionbox{The wave after 125 time steps. It has move almost across the string.}{\includegraphics[scale = 0.3]{opp8t125M1.png}}
\par
\subcaptionbox{The wave after 200 time steps. The wave has just been reflected of the right wall.}{\includegraphics[scale = 0.3]{opp8t200M1.png}}
\subcaptionbox{The wave after 400 time steps. The wave has just been reflected of the left wall.}{\includegraphics[scale = 0.3]{opp8t400M1.png}}
\caption{A triangular wave moving to the right using $\Delta t = \sqrt{\frac{m}{k}}$. The triangle holds its shape, moves to the right, reflects off the right wall, moves back and reflects off left wall. As we can see the wave behaves exactly as we expects.}
\end{figure}

We can see from the above figure that the wave moves how it suppose to. This method of making the wave move to the right does not depend on the size or initial condition, so it is very versatile, and very exact. The only problem is the set size of $\Delta t$. If we for some reason want to lower the time step, we have to use the interpolation. But as we will see below this creates imperfections in the simulation.

\subsection*{Results Method 2:}


\begin{figure}[H]
\captionsetup[subfigure]{position=b}
\centering
\subcaptionbox{The initial triangular wave.}{\includegraphics[scale = 0.3]{opp8t0M2.png}}
\subcaptionbox{The wave after 250 time steps. It has move almost across the string.}{\includegraphics[scale = 0.3]{opp8t250M2.png}}
\par
\subcaptionbox{The wave after 400 time steps. The wave has just been reflected of the right wall.}{\includegraphics[scale = 0.3]{opp8t400M2.png}}
\subcaptionbox{The wave after 800 time steps. The wave has just been reflected of the left wall.}{\includegraphics[scale = 0.3]{opp8t800M2.png}}
\caption{A triangular wave moving to the right using interpolation. We can see that the moving wave is not a perfect triangle. The triangle has after the initial time steps no longer amplitude 1, and there are ripples following it. But other than that, the wave behaves as we expect. This is simulated with $\Delta t = 0.5\cdot \sqrt{\frac{m}{k}}$}
\end{figure}

As we see in the figure above, the triangle moves to the right, reflects of the right wall. And due to the near infinite impedance of the end point its amplitude changes sign. The reflected wave moves back and reflects of the left wall. We can see that we have a amplitude less than the original, and ripples behind the wave. This is due to us trying to impose continuity on a discrete system. We can look at the initial and the pre-initial wave:

\begin{figure}[H]
\centering
\includegraphics[scale=0.4]{opp8init.png}
\caption{The initial and pre-initial triangle. They are almost identical (and shifted) but for the top and the endpoints.}
\end{figure}

We can see that because we are interpolate, the top of the triangle is missing(it only exist if the mass point had been $1\cdot v\Delta t$ away), and endpoints are a kink. These are the reasons we are getting the imperfections in the triangular wave.\\

But other then the above mentioned effects, the triangle wave behaves as we expected. Moving to the right and reflecting of the walls.









\appendix

\section{Code for exercise 5:}
\lstinputlisting{opp5.py}

\section{Code for exercise 6:}
\lstinputlisting{opp6.py}

\section{Code for exercise 8:}
\lstinputlisting{opp8.py}

\end{document}

